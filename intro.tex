\newpage
\sectionnonumber{Введение}

\subsection*{Актуальность темы}
Интернет может нести не только пользу, но также и вред. Одна из его опасностей — фиктивные аккаунты. Они существуют как в различных социальных сетях, так и на других платформах; некоторые используются в безобидных целях, а другие — для распространения ложной информации.

<<Фейки>> могут быть созданы с разными целями: получить коммерческую выгоду, дискредитировать настоящего пользователя, заполучить личную информацию и так далее. Это очень серьезная опасность, которую не всегда можно распознать с первого взгляда. Некоторые пользователи мировой сети не догадываются, кто скрывается в диалоге с интернет-знакомым — люди могут умело подделывать аккаунты в социальных сетях. Кроме того, часто бывает, что за страницей обычных пользователей могут скрываться автоматизированные программы. Они могут практически не отличаться от обычных аккаунтов.

Особенно сильно это наносит вред такой среде, как наука. Подделка информации в этой сфере несет огромный вред, который может распространяться на все общество. Научные данные используются везде: от строительства домов, до лечения животных, и неточности или ошибки в них могут стоить дорого. 

Основная проблема заключается в том, что фиктивные аккаунты, которые создаются в системах, связанных с наукой, тяжело отличимы от обычных аккаунтов. Для поддержания стабильной работы публикации статей необходимо быстрое реагирование на <<фейки>> и их оперативное удаление. 

\subsection*{Цель и задачи}
Целью курсовой работы является реализация программной системы, выявляющей фиктичные аккаунты в системе Open Journal Systems. Для достижения поставленной цели необходимо решить следующие задачи:
\begin{enumerate}
	\item Провести анализ предметной области и литературы по теме работы.
	\item Разработать алгоритм выявления фиктивных аккаунтов.
	\item Спроектировать интерфейс программной системы и модульной структуры приложения.
	\item Реализовать программную систему, выявляющую фиктивные аккаунты, на основе разработанного алгоритма.
        \item Подготовить набор тестов, выполнение тестирования программной системы.
\end{enumerate}

\subsection*{Структура и содержание работы}
Написать разделы, из которых состоит курсовая работа.