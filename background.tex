\newpage
\section{Анализ предметной области}
\label{sec:Background}

В разделе 1.1. располагается описание предметной области. В разделе 1.2. рассматриваются аналоги и существующие решения.
\vspace{1.5em}

\subsection{Описание предметной области}
Open Journal System — это программное обеспечение, которое позволяет публиковать статьи и организовать рабочий процесс издательства. На ее основе разработаны многие порталы, работают институты, научные центры и журналы в разных странах мира (интерфейс OPS переведён более чем на 30 языков). Платформа обладает модульной структурой и имеет возможность подключения плагинов. 

OJS может быть рассмотрена как электронная библиотека: программа обеспечивает доступ к контенту, поиск по нему (автора, ключевые слова, названия статей, год выпуска и так далее). 
\vspace{1.5em}

\subsection{Обзор литературы}
Для выбора наилучшего подхода к решению задач и достижения цели были прочитаны некоторые статьи на тему выявления <<фейков>>. Исследования, посвящённые поиску фиктивных аккаунтов, представлены в следующих работах:

\begin{enumerate}
    \item В работе <<Fake Accounts Identification in Mobile Communication 
    Networks Based on Machine Learning>>~\cite{HassanAA23} раскрывается проблема поддельных страниц. Их количество растет вместе с увеличением числа активных пользователей социальных сетей. Поддельные профили на сайтах социальных сетей создают ненастоящие новости и распространяют нежелательные материалы, содержащие спам-ссылки. 

    В этой статье приводится контролируемый алгоритм машинного обучения, называемый машиной опорных векторов (SVM), который используется вместе с методом случайного леса. Эту концепцию можно легко применить для идентификации миллионов учетных записей, которые невозможно проверить вручную. Данная модель сравнивается с другими методами идентификации, и результаты показывают, что предложенный алгоритм работает с большей точностью. 

    \item Статья <<Social Networks Fake Profiles Detection 
    Based on Account Setting and Activity>>~\cite{ElyusufiEK19} посвящена выявлению поддельных профилей в социальных сетях. Подходы к выявлению поддельных профилей в социальных сетях можно разделить на подходы, направленные на анализ данных профилей. Для обнаружения поддельных профилей было предложено множество алгоритмов и методов. В этой статье оценивается влияние использования дерева решений (DT) и наивного алгоритма Байеса (NB) для классификации профилей пользователей на поддельные и подлинные.

    \item В работе <<Novel approaches to fake news and fake account detection in OSNs: user social engagement and visual content centric model>>~\cite{UppadaMVHS22} исследователи используют модель SENAD, которая определяет подлинность новостных статей, публикуемых в Twitter, на основе подлинности и предвзятости пользователей, которые взаимодействуют с этими статьями. Предлагаемая модель включает в себя идею оценки соотношения подписчиков, возраст аккаунта и т.д. Для анализа изображений предлагается использовать нейронную сеть (CredNN). Предложенная гибридная идея объединения ELA и Sent и анализа настроений играет важную роль в обнаружении поддельных изображений с точностью около 76\%.
    
    \item В исследовании <<Fake Twitter accounts: Profile characteristics obtained using an activity-based pattern detection approach>>~\cite{GurajalaWHM15} проведен анализ 62 миллионов общедоступных профилей пользователей Twitter, и разработана стратегия идентификации автоматически сгенерированных поддельных профилей. Используя алгоритм сопоставления шаблонов имен, анализ времени обновления твитов и времени создания профилей, было выявлено множество фиктивных учетных записей пользователей.
\end{enumerate}

