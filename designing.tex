\newpage
\section{Проектирование}
\label{sec:designing}
В разделе 2.1. описаны варианты использования системы. В разделе 2.2. представлены функциональные и нефункциональные требования к системе. В разделе 2.3. показаны зарисовки графического интерфейса.

\vspace{1.5em}
\subsection{Варианты использования системы}
\label{subsec:Variants}
Описание способов взаимодействия с системой, кто с ней может работать и каким образом отображено на рис.~\ref{ris:variantsUse}.

\begin{figure}[!ht]
    \center{\includegraphics[width=1\linewidth]{image}}
    \includegraphics[scale=0.6]{Курсовая работа/pic/Варианты использования системы.png}
    \medskip
    \caption{Диаграмма вариантов использования}
    \label{ris:variantsUse}
\end{figure}

Единственный актер, который может взаимодействовать с системой распознавания фиктивных аккаунтов в OJS – это пользователь. Он может выполнять следующие действия:

\begin{enumerate}
    \item Загрузить данные системы. Пользователь совершает выгрузку данных об аккаунтов.
    \item Выбрать алгоритм для нахождения фиктивных аккаутов. Пользователь выбирает подходящий метод, который будет искать аномалии в виде поддельных учетных записей.
    \item Выбрать признаки, по которым происходит кластеризация. Пользователь отмечает признаки, по которым необходимо отыскивать фиктивные аккаунты.
    \item Настроить параметры алгоритма. Пользователь настраивает параметры выбранного метода.
    \item Добавить отображение результатов кластеризации. Пользователь отображает результаты кластеризации после работы алгоритма.
    \item Удалить фиктивные аккаунты. Пользователь совершает удаление всех аккаунтов, отмеченных как фиктивные, или выборочно некоторые из них.
\end{enumerate}

\subsection{Требования к системе}
\label{subsec:fotmalDefinition}
\textbf{Функциональные требования}

Функциональные требования определяют функциональность программного обеспечения, то есть описывают, какое поведение должна предоставлять разрабатываемое приложение

\textbf{Нефункциональные требования}

К нефункциональным требованиям системы относятся свойства, которыми она должна обладать. Например, удобство использования, безопасность, расширяемость и т.д.

\vspace{1.5em}
\subsection{Графический интерфейс}
\label{subsec:Graphic}
Представление графического интерфейса программной системы.
