\newpage
\section{Проектирование}
\label{sec:designing}
В разделе 2.1. представлены функциональные и нефункциональные требования к системе. В разделе 2.2. описаны варианты использования системы. В разделе 2.3. описывается предобработка данных. а в разделе 2.4. представлена архитектура приложения, а в разделе 2.5 показан графический интерфейс. 

\vspace{1.5em}
\subsection{Требования к системе}
\label{subsec:fotmalDefinition}
\textbf{Функциональные требования}

Функциональные требования определяют функциональность программного обеспечения, то есть описывают, какое поведение должна предоставлять разрабатываемое приложение

\textbf{Нефункциональные требования}

К нефункциональным требованиям системы относятся свойства, которыми она должна обладать. Например, удобство использования, безопасность, расширяемость и т.д.

\vspace{1.5em}
\subsection{Варианты использования системы}
\label{subsec:Variants}
Описание способов взаимодействия с системой, кто с ней может работать, каким образом.

\vspace{1.5em}
\subsection{Архитектура приложения}
\label{subsec:Architecture}
Показать модули приложения, расписать то, что делает каждый из них.

\vspace{1.5em}
\subsection{Графический интерфейс}
\label{subsec:Graphic}
Представление графического интерфейса программной системы.